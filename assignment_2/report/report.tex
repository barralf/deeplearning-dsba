%++++++++++++++++++++++++++++++++++++++++
% Don't modify this section unless you know what you're doing!
\documentclass[10pt, a4paper]{article}


\usepackage[latin1]{inputenc} %g�re les caract�res accentu�s et symboles sp�cifiques de l?alphabet d?une langue
\usepackage[T1]{fontenc}%gere les polices de caract�res (definition de la position des caract�res dans une police)
\usepackage{lmodern}%police vectorielle
\usepackage{hyperref}%transforme toutes les r�f�rences internes du document en hyperliens
\usepackage{fancyhdr}%pour des ent�tes am�lior�s
\usepackage{lastpage}%pour pouvoir utiliser 
\usepackage{xcolor,graphicx}%pour des documents en couleur et ins�rer des images
\usepackage{pgf,tikz}%package permettant d?inclure des figures au format PDF en restant dans l?environnement LATEX.
\usepackage{pgfplots}%pour tracer des courbes/graphiques
\usepackage{framed}%minipages avec cadre autour
\bibliographystyle{ieeetr}%style de bibliographie
\usetikzlibrary{arrows}%dessin de fleches
\usepackage{caption}%permet de customiser les l�gendes dans des environnements flottants
\usepackage{amsmath,amssymb,color,bm}%charge le formulaire mathematique+symb+caracteres gras  avec \boldsymbol en restant un caract mathematique en italique
\usepackage{ulem}
\usepackage{layout}%permet de tracer la mise en page avec \layout
\usepackage{cite} % takes care of citations
\newcommand*{\Coord}[3]{% 
  \ensuremath{\overrightarrow{#1}\, 
    \begin{pmatrix} 
      #2\\ 
      #3 
    \end{pmatrix}}}

%miseenpage
\setlength{\oddsidemargin}{0cm}
\setlength{\topmargin}{0cm}
\setlength{\textheight}{24.7cm}
\setlength{\marginparwidth}{0cm}
\setlength{\textwidth}{15.8cm}
\setlength{\marginparsep}{0pt}
\setlength{\voffset}{-1cm}
\setlength{\hoffset}{-0.29cm}
\setlength{\headheight}{14.5pt} %pas vraiment n�c�ssaire mais cr�e des warnings sans...

%\setlength{\parindent}{0pt} %indentation


\setlength{\oddsidemargin}{0cm}
\setlength{\topmargin}{-1cm}
\setlength{\textheight}{24cm}
\setlength{\marginparwidth}{0.2cm}
\setlength{\textwidth}{16cm}
\setlength{\voffset}{-0.5cm}
\setlength{\headheight}{14.5pt} %pas vraiment n�c�ssaire mais cr�e des warnings sans...
%++++++++++++++++++++++++++++++++++++++++


\begin{document}

\title{Mini Project 2 - Deep Learning}
\author{Florian \textsc{Barral}}

\date{\today}
\maketitle

%\begin{abstract}
%\end{abstract}

\section*{Question 2}

\begin{align*}
& W^*  = \underset{W \in \mathcal{O}_d(\mathbb{R})}{\mathrm{argmin}}|| WX-Y||_F \\
\iff & W^* =  \underset{W \in \mathcal{O}_d(\mathbb{R})}{\mathrm{argmin}} \langle WX-Y | WX-Y \rangle \\
\iff & W^* =  \underset{W \in \mathcal{O}_d(\mathbb{R})}{\mathrm{argmin}} ||WX||^2 + ||Y||^2 -2 \langle WX | Y \rangle \\
\iff & W^* =  \underset{W \in \mathcal{O}_d(\mathbb{R})}{\mathrm{argmax}} \langle WX | Y \rangle \\
\iff & W^* =  \underset{W \in \mathcal{O}_d(\mathbb{R})}{\mathrm{argmax}} \langle W | YX^T \rangle \\
\iff & W^* =  \underset{W \in \mathcal{O}_d(\mathbb{R})}{\mathrm{argmax}}  \langle W | U\Sigma V^T \rangle \\
\iff & W^* =  \underset{W \in \mathcal{O}_d(\mathbb{R})}{\mathrm{argmax}}  \langle U^T W V | \Sigma  \rangle \\
\iff & W^* =  U \underbrace{\underset{W' \in \mathcal{O}_d(\mathbb{R})}{\mathrm{argmax}}  \langle  W' | \Sigma  \rangle}_{Id}  V^T\\
\iff & W^* =  UV^T\\
\end{align*}
\end{document}

%\mathbf M=(M_{i,j})_{(i,j) \in{[\![ 1,n]\!]\times[\![ 1,m]\!]}},~~~  \mathbf {MM^T}=((MM^T)_{i,j})_{(i,j) \in{[\![ 1,n]\!]^2}},~~~  \mathbf {M^TM}=((M^TM)_{i,j})_{(i,j) \in{[\![ 1,n]\!]^2}}
% \begin{align*}
%\forall (i,j) \in [\![ 1,n]\!]^2, ~~~~~
%((MM^T)_{i,j} & = \sum_{k=1}^{m} M_{i,k} M^T_{k,j} \\
%& = \sum_{k=1}^{m} M_{i,k} M_{j,k} 
%\end{align*}

%\begin{align*}
%\forall (i,j) \in [\![ 1,n]\!]^2, ~~~~~
%(M^TM)_{i,j} & = \sum_{k=1}^{m} M^T_{i,k} M_{k,j} \\
%& = \sum_{k=1}^{m} M_{k,i} M_{k,j} 
%\end{align*}